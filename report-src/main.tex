\documentclass[11pt, a4paper]{article}
\usepackage[utf8]{inputenc}
\usepackage[margin=1in]{geometry} % Set 1-inch margins
\usepackage{graphicx}


\title{AA Final Exam}
\author{Andreas Riget Bagge \& Marcus Skjold Pedersen}
\date{2024-12-20}

\begin{document}
\maketitle

\section{Part 1}

\subsection{Task 1}
We have implemented classical MergeSort recursively in the class \verb|TopDownMergeSort|, largely based on the "Top-down MergeSort" algorithm from \cite{Sedgewick_Wayne_2011}. The algorithm takes an array \verb|a[lo..hi]| and recursively divide it up into subarrays \verb|[a..mid]+[mid+1..hi]| for $mid=lo+(hi-lo)/2$, stopping at subarrays of length 1. It then merges the subarrays back together pairwise, using class \verb|Merge|, to sorted subarrays till the whole array is merged back together.

The merging works by putting the subarrays into an auxiliary array of, \verb|aux|, then merging them back together by iterating through the subarrays from lowest to highest element, repeatedly comparing the current element from each subarray and picking the lowest one. In case of a tiebreak, we pick from the left subarray, in order to ensure that the sort is \textit{stable}. After the final merge the sum of compares for all merges, corresponding to the compares performed during the sort, is returned.

Tests are done in \verb|TopDownMergeSortTest|. Most of the tests for this class are relevant for subsequent implementations and, as such, are implicitly included for those as well. We have tested for edge-cases such as arrays of length $\leq 1$, already sorted arrays and inverted arrays. Further tests has been performed for this implementation on arrays of variable sizes to ensure that the correct number of compares are returned, and that they are correctly sorted. We have also made tests on arrays with duplicate elements to ensure stability of the sorting. For this we have made a \verb|TestData| class with an id-field and a value-field, sorting based on the value-field and checking that the order for the id-fields for equal elements are maintained.

As the merging is common to more implementation, it is given its own class, and tested for itself. These tests mostly corresponds to tests performed on our merge-sort-implementations, but we especially focus on cases where sub-arrays of even and differing sizes are merged, corresponding to cases where Mergesort divides up arrays unevenly.

\subsection{Task 2}

\subsection{Task 3}

We have implemented a version of recursive mergesort with a base-case of insertion-sort for a parameter \verb|c| in \verb|TopDownMergeSortCutoff|. This version is very similar to our \textit{Task 1}-implementation, but rather than stopping the recursion for subarrays of size $\leq 1$, it stops at subarrays of sizes $c$ or less, ensured with the check \verb|hi <= lo + c - 1|, and switches to Insertion-sort for the subarray instead.\footnote{As such, running the cutoff-variant (as well as the one in Task6) with c=1, corresponds to having no cutoff-value. However, to keeps tasks separate, we keep the two variants in different classes}

We have implemented Insertion-sort in class \verb|Insertion-sort|, also inspired by the variant in \cite{Sedgewick_Wayne_2011}. 

We have tested both insertion sort and our cutoff-variant in test-classes corresponding to their names. For the cutoff-variant, we have additionally checked that it correctly handles cases where $c\leq 0$, throwing an error rather than looping indefinitely, and that it just performs insertionsort for $c\geq$ \verb|a.length|. Different cutoff-values has also been tried, also ones where subarrays of length $< c$ has been sorted with insertion-sort.  

\subsection{Task 4}

\subsection{Task 5}

We have implemented an iterative version of merge sort using a stack in \verb|BottomUpMergeSort|. This implementation iterates through the input array, \verb|a|, adding each element as a run of length 1, to a stack of computed \textit{runs}, i.e. already sorted sub-arrays of the input. Then, if the next element in the stack has the same length as the newly computed run, they are merged. This is repeated till this condition no longer holds. 
When \verb|a| has been iterated through, the runs of the stacks are merged together, one by one, till whole array is merged back together.

Following this procedure the stack invariantly holds runs of length $2^n, n\in \mathbf{N}$, and the length of runs in the stack must be strictly decreasing. Owing to this invariant, we have modeled the stack as the binary representation of an \verb|int|, where a run of length $2^n$ is represented as the $n$'th bit being set in the binary number. As the value of the binary number corresponds to the sum of run-lengths in the stack, the positions of current runs can be calculated from the value of the stack and the length of the run currently being merged. The same holds true for subsequently merging all the runs in the stack into one. 

The implementation is tested similarly to the merge-sort implementations of Tasks 1 and 3. However, we have also made sure to both cover cases where there are no "gaps" between computed runs in the resulting runstack (a size 7 array results in the run-stack being 111 in binary), cases with just one run in resulting run-stack (size 4 array giving 100), and ones with gaps between runs (size 11 giving 1011).

\subsection{Task 6}

We have made a variant of the algorithm in task 5, \verb|BottomUpMergeSortCutoff|, that makes new runs of size $c$ (using insertion sort) for a given integer $c\geq1$, rather than size 1. If less than $c$ elements remain in the array, insertion-sort is used on the remaining elements. 

This variant is a lot similar to the implementation of Task 5. However, as the length of runs now won't necessarily be a factor of 2, but will still double after each merge, we keep representing the stack as a binary number, but multiplies with $c$ when finding indexes for merging. Also the array is iterated through in intervals of $c$, making runs with insertion-sort. $>c$ residues are computed with insertion-sort, and the final merge is then done similarly to Task 5.

We have tested the implementation similarly to our implementation of task 5, adding the special cases mentioned for the cutoff-variant in task 3. We have also made sure especially to test for cases where the last computed run are of length $<c$ and one for the opposite case. ***

\subsection{Task 7}


\section{Part 2}

\subsection{Task 8}

\subsection{Task 9}

\subsection{Task 10}

\section{Part 3}

\subsection{Task 11}

\subsection{Task 12}

\subsection{Task 13}

\subsection{Task 14}

\subsection{Task 15}

\subsection{Task 16}

\subsection{Task 17}


\section{Introduction}

Introduce the problem you are addressing here, and give a very brief
overview of what you are about to report on.

\section{Implementation}

Explain briefly what you have implemented and how. This might also be
an appropriate place to explain how you verified the correctness of
your implementation.

\section{Experiments}

Explain what kind of experiments you run. Also tell what kind of
hardware and software were used to run the experiments.

Present your results numerically in a table, and also visually as a
plot. Remember to refer to the tables and figures in your text; it is
not sufficient to only include them, but they should also be made part
of the prose. Like so: here we refer to Table~\ref{tbl:resultscubic}.

\begin{table}[h]
  \begin{center}
  \caption{Write a caption that tells what this table is about.}
  \label{tbl:resultscubic}
  % Uncomment the line below to include the automatically generated
  % table from the file.
  % \input{threesum_cubic_tabular.tex}
  \end{center}
\end{table}

\begin{figure}[h]
  \begin{center}
    % uncomment the line below to include the plot that you
    % automatically generated.
    % \includegraphics[width=\textwidth]{plot_cubic_vs_hashmap.pdf}
    \caption{Write a descriptive caption here.}
    \label{fig:runtimes}
  \end{center}
\end{figure}

\newpage
\bibliographystyle{plain}
\bibliography{misc/cite}

\end{document}